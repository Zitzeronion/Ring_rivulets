%%%%%%%%%%%%%%%%%%%%%%%%%%%%%%%%%%%
%This is the LaTeX ARTICLE template for RSC journals
%Copyright The Royal Society of Chemistry 2016
%%%%%%%%%%%%%%%%%%%%%%%%%%%%%%%%%%%

\documentclass[twoside,twocolumn,9pt]{article}
\usepackage{extsizes}
\usepackage[super,sort&compress,comma]{natbib} 
\usepackage[version=3]{mhchem}
\usepackage[left=1.5cm, right=1.5cm, top=1.785cm, bottom=2.0cm]{geometry}
\usepackage{balance}
\usepackage{mathptmx}
\usepackage{amsmath}
\usepackage{amssymb}
%\usepackage{amssymb}
\usepackage{sectsty}
\usepackage{graphicx} 
\usepackage{lastpage}
\usepackage[format=plain,justification=justified,singlelinecheck=false,font={stretch=1.125,small,sf},labelfont=bf,labelsep=space]{caption}
\usepackage{float}
\usepackage{fancyhdr}
\usepackage{fnpos}
% \usepackage{cuted}
\usepackage[english]{babel}
\addto{\captionsenglish}{%
  \renewcommand{\refname}{Notes and references}
}
\usepackage{array}
\usepackage{droidsans}
\usepackage{charter}
\usepackage[T1]{fontenc}
\usepackage[usenames,dvipsnames]{xcolor}
\usepackage{setspace}
\usepackage[compact]{titlesec}
\usepackage{hyperref}
%%%Please don't disable any packages in the preamble, as this may cause the template to display incorrectly.%%%

\usepackage{epstopdf}%This line makes .eps figures into .pdf - please comment out if not required.

\definecolor{cream}{RGB}{222,217,201}
\definecolor{jlblue}{rgb}{0.0,0.6056031611752245,0.9786801175696073}
\definecolor{jlorange}{rgb}{0.8888735002725198,0.43564919034818994,0.2781229361419438}
\definecolor{jlgreen}{rgb}{0.2422242978521988,0.6432750931576305,0.3044486515341153}
\definecolor{jlviolet}{rgb}{0.7644401754934356,0.4441117794687767,0.8242975359232758}

\begin{document}

\pagestyle{fancy}
\thispagestyle{plain}
\fancypagestyle{plain}{
%%%HEADER%%%
\renewcommand{\headrulewidth}{0pt}
}
%%%END OF HEADER%%%

%%%PAGE SETUP - Please do not change any commands within this section%%%
\makeFNbottom
\makeatletter
\renewcommand\LARGE{\@setfontsize\LARGE{15pt}{17}}
\renewcommand\Large{\@setfontsize\Large{12pt}{14}}
\renewcommand\large{\@setfontsize\large{10pt}{12}}
\renewcommand\footnotesize{\@setfontsize\footnotesize{7pt}{10}}
\makeatother

\renewcommand{\thefootnote}{\fnsymbol{footnote}}
\renewcommand\footnoterule{\vspace*{1pt}% 
\color{cream}\hrule width 3.5in height 0.4pt \color{black}\vspace*{5pt}} 
\setcounter{secnumdepth}{5}

\makeatletter 
\renewcommand\@biblabel[1]{#1}            
\renewcommand\@makefntext[1]% 
{\noindent\makebox[0pt][r]{\@thefnmark\,}#1}
\makeatother 
\renewcommand{\figurename}{\small{Fig.}~}
\sectionfont{\sffamily\Large}
\subsectionfont{\normalsize}
\subsubsectionfont{\bf}
\setstretch{1.125} %In particular, please do not alter this line.
\setlength{\skip\footins}{0.8cm}
\setlength{\footnotesep}{0.25cm}
\setlength{\jot}{10pt}
\titlespacing*{\section}{0pt}{4pt}{4pt}
\titlespacing*{\subsection}{0pt}{15pt}{1pt}
%%%END OF PAGE SETUP%%%

%%%FOOTER%%%
\fancyfoot{}
\fancyfoot[LO,RE]{\vspace{-7.1pt}\includegraphics[height=9pt]{head_foot/LF}}
\fancyfoot[CO]{\vspace{-7.1pt}\hspace{13.2cm}\includegraphics{head_foot/RF}}
\fancyfoot[CE]{\vspace{-7.2pt}\hspace{-14.2cm}\includegraphics{head_foot/RF}}
\fancyfoot[RO]{\footnotesize{\sffamily{1--\pageref{LastPage} ~\textbar  \hspace{2pt}\thepage}}}
\fancyfoot[LE]{\footnotesize{\sffamily{\thepage~\textbar\hspace{3.45cm} 1--\pageref{LastPage}}}}
\fancyhead{}
\renewcommand{\headrulewidth}{0pt} 
\renewcommand{\footrulewidth}{0pt}
\setlength{\arrayrulewidth}{1pt}
\setlength{\columnsep}{6.5mm}
\setlength\bibsep{1pt}
%%%END OF FOOTER%%%

%%%FIGURE SETUP - please do not change any commands within this section%%%
\makeatletter 
\newlength{\figrulesep} 
\setlength{\figrulesep}{0.5\textfloatsep} 

\newcommand{\topfigrule}{\vspace*{-1pt}% 
\noindent{\color{cream}\rule[-\figrulesep]{\columnwidth}{1.5pt}} }

\newcommand{\botfigrule}{\vspace*{-2pt}% 
\noindent{\color{cream}\rule[\figrulesep]{\columnwidth}{1.5pt}} }

\newcommand{\dblfigrule}{\vspace*{-1pt}% 
\noindent{\color{cream}\rule[-\figrulesep]{\textwidth}{1.5pt}} }

\makeatother
%%%END OF FIGURE SETUP%%%

%%%TITLE, AUTHORS AND ABSTRACT%%%
\twocolumn[
  \begin{@twocolumnfalse}
{\includegraphics[height=30pt]{head_foot/SM}\hfill\raisebox{0pt}[0pt][0pt]{\includegraphics[height=55pt]{head_foot/RSC_LOGO_CMYK}}\\[1ex]
\includegraphics[width=18.5cm]{head_foot/header_bar}}\par
\vspace{1em}
\sffamily
\begin{tabular}{m{4.5cm} p{13.5cm} }

\includegraphics{head_foot/DOI} & \noindent\LARGE{\textbf{Instabilities of ring-rivulets: Impact of substrate wettability$^\dag$}} \\%Article title goes here instead of the text "This is the title"
\vspace{0.3cm} & \vspace{0.3cm} \\

 & \noindent\large{Stefan Zitz,\textit{$^{a\ast}$} Andrea Scagliarini,\textit{$^{b,c}$} Johan Roenby\textit{$^{a}$}} \\%Author names go here instead of "Full name", etc.

\includegraphics{head_foot/dates} & \noindent\normalsize{
\noindent Rivulets and droplets are naturally appearing shapes when small amounts of liquid are deposited on a partially wettable substrate.
Here we study, by means of numerical simulations, the dewetting dynamics of a ring-rivulet on substrates with various contact angles and wettability patterns.
In particular, we consider, a base case with homogeous contact angle, a case with an annular band of lower contact angle as compared to the background, and cases with a constant positive and negative radial contact angle gradients. 
We show that by tuning the parameters characterizing the patterns, it is possible to control 
not only the stability of the rivulet, i.e. its breakup/collapse dynamics and 
the associated time scales, but also the dewetting morphology, in terms of number and position of the formed droplets.
%Thus making it possible to enhance the probability for a Rayleigh-Plateau type %break up into multiple droplets or to accelerate the capillary retraction that %stems from the curvature difference of the ring and leads to a single droplet.
}

\end{tabular}

 \end{@twocolumnfalse} \vspace{0.6cm}

]
%%%END OF TITLE, AUTHORS AND ABSTRACT%%%

%%%FONT SETUP - please do not change any commands within this section
\renewcommand*\rmdefault{bch}\normalfont\upshape
\rmfamily
\section*{}
\vspace{-1cm}


%%%FOOTNOTES%%%

\footnotetext{\textit{$^{a}$~IMFUFA, Department of Science and Environment, Roskilde University, Postbox 260, 4000 Roskilde, DK. Tel: +45 2993 1923; E-mail: johan@ruc.dk}}
\footnotetext{\textit{$^{\ast}$~current E-mail: stefan.zitz@rcpe.at}}

\footnotetext{\textit{$^{b}$~Institute for Applied Mathematics "M. Picone" (IAC), Consiglio Nazionale delle Ricerche (CNR), Via dei Taurini 19, 00185 Rome, Italy, E-mail: andrea.scagliarini@cnr.it}}
\footnotetext{\textit{$^{c}$~INFN, sezione Roma ``Tor Vergata'', via della Ricerca Scientifica 1, 00133 Rome, Italy}}


%Please use \dag to cite the ESI in the main text of the article.
%If you article does not have ESI please remove the the \dag symbol from the title and the footnotetext below.
\footnotetext{\dag~Electronic Supplementary Information (ESI) available: \href{https://github.com/Zitzeronion/Ring_rivulets}{Ring\_rivulets}. See DOI: 10.1039/cXsm00000x/}
%additional addresses can be cited as above using the lower-case letters, c, d, e... If all authors are from the same address, no letter is required

%\footnotetext{\ddag~Additional footnotes to the title and authors can be included \textit{e.g.}\ `Present address:' or `These authors contributed equally to this work' as above using the symbols: \ddag, \textsection, and \P. Please place the appropriate symbol next to the author's name and include a \texttt{\textbackslash footnotetext} entry in the the correct place in the list.}


%%%END OF FOOTNOTES%%%

%%%MAIN TEXT%%%%
\section{Introduction}
\label{sec:intro}
Thin liquid films and droplets are widespread in our every day life and play a crucial role in a host of natural and technological applications, from painting and coating to lab-on-a-chip devices to biofluidics~\cite{degennesCapillarityWettingPhenomena2004, ronsinPhaseFieldSimulationsMorphology2022,fockeLabonaFoilMicrofluidicsThin2010}.
Understanding their dynamics and controlling their stability is, therefore, a central problem for applied research in process engineering and nanotechnology~\cite{singhInkjetPrintingProcess2010, quereFluidCoatingFiber1999, utadaDrippingJettingDrops2007}, but also poses fundamental questions lying at the crossroads between fluid dynamics and chemical physics~\cite{oronLongscaleEvolutionThin1997, beckerComplexDewettingScenarios2003, thielePatternedDepositionMoving2014, wilczekSlidingDropsEnsemble2017, peschkaSignaturesSlipDewetting2019}.
Dewetting induced by intrinsic instabilities of the film and/or impurities on the substrate surface, for instance, can undermine the effectiveness of a coating process~\cite{bonnWettingSpreading2009, chenWrinklingInstabilitiesPolymer2012}. 
On the other hand, breakup of deposited structures such as rivulets is exploited in the generation of droplets {\it on demand}~\cite{}.
All these phenomena involve inherently multiscale problems, that span from the molecular motion at the three phase contact line to the nano-/mircoscale thickness of the film up to the macroscopic area the coating covers, thus posing non-trivial computational challenges.
The dewetting of a fluid rivulet deposited on a substrate recalls the classical fluid dynamic problem of a filament breakup, driven by the Rayleigh-Plateau instability, with the additional complexity of the fluid-solid physico-chemical interactions~\cite{diezBreakupFluidRivulets2009, diezStabilityFinitelengthRivulet2009, diezInstabilityTransverseLiquid2012}.
Recently,  a number of experimental and theoretical/numerical studies have focused on ring-shaped rivulets, that were shown to be useful precursors of droplet patterns with a circular symmetry~\cite{nguyenCompetitionCollapseBreakup2012, gonzalezStabilityLiquidRing2013, wuCompetingLiquidPhase2011, edwardsControllingBreakupToroidal2021}. 
Here, at difference with the simpler straight rivulet case, the dewetting dynamics depends also on the non-uniform curvature of the rivulet ringed shape, and in particular on the different curvatures of the two contact lines.
%Recently, Suo et al.~\cite{suoDewettingCornerFilm2023} have studied a related problem where the film admits a %wedge shape and a solid cylinder fills the inner part of the ring. 
%The ring is therefore not allowed to contract, but can breakup and form droplets.
%They further show a good agreement between theory and simulations as well as a clear dependence of the %dynamics on the initial width of the film and the contact angles at each boundary.
Self- and direct assembly of nanomaterials from liquid nanostructures has been one of the driving forces for the study of Nguyen et al.~\cite{nguyenCompetitionCollapseBreakup2012} and earlier studies of Wu et al.~\cite{wuBreakupPatternedNanoscale2010} where they showed that liquid-metal rings are suited to form arrays of droplets.
Diez et al.~\cite{diezBreakupFluidRivulets2009, diezStabilityFinitelengthRivulet2009} laid the theoretical foundation for the stability of a straight rivulet in their work using linear stability analysis (LSA) and numerical simulations. 
Later, Gonz{\'a}lez et al.~\cite{gonzalezStabilityLiquidRing2013} extended these results to ring rivulets, determining the characteristic time scales for collapse and breakup and showing that the main control parameter discriminating between the two instability routes is the rivulet aspect-ratio, namely the ratio of its width over the its radius.
They also provided predictions on the expected number of formed droplets as dictated by the most unstable wavelength.
An interesting question that may naturally arise is whether and to which extent it is possible to further control the fate of ring rivulets and the consequent dewetting morphologies by properly treating the substrate to exploit wettability patterns. 
With recent developments in surface chemistry and the emerging technology of switchable substrates local precise wettability gradients are now readily attainable~\cite{xinReversiblySwitchableWettability2010, stuartEmergingApplicationsStimuliresponsive2010,chenThermalresponsiveHydrogelSurface2010, ichimuraLightDrivenMotionLiquids2000, mugeleElectrowettingConvenientWay2005, edwardsControllingBreakupToroidal2021}.
While a number of studies have focused on thin film dewetting and droplet transport on patterned substrates~\cite{liuActuatingWaterDroplets2015,grawitterSteeringDropletsSubstrates2021, zitzControllingDewettingMorphologies2023}, the case of ring-shaped rivulets remained so far almost unexplored.
A relevant exception is represented by the work of Edwards et al~\cite{edwardsControllingBreakupToroidal2021}, who have studied numerically and experimentally liquid rings deposited on a substrate where the contact angle was controlled by electrowetting. 
They showed that different electric potentials can be used not only to control the number of droplets after breakup but also to fully reverse the process.

In the present paper, we perform a systematic study of the effect of the ring rivulet initial geometry (width over radius ratio) and of two types of wettability patterns in the selection of route towards either retraction and collapse to a single droplet or towards the breakup into multiple droplets. 
We show that by depositing the liquid ring onto an annular region of lower contact angle (with respect to the background substrate) one can remove the collapse mode. Moreover, the contact angle contrast turns out to serve as a control parameter determining together with the dimensionless initial ring width the number of droplets in the final stationary equilibrium state.
By introducing, instead, a radially symmetric linear contact angle profile, coaxial with the ring and pointing either inwards or outwards, we find that it is possible, in the former case, to control the retraction speed while steering the number of metastable droplets whereas, in the latter case, to stabilize the ring rivulet against collapse.
The outline of the paper is as follows: In the next section, Sec.~\ref{sec:method} we introduce the method we use to run numerical experiments.
We then present our results in Sec.~\ref{sec:dynamics}, starting with a comparison with the literature and then present the impact of the wettability patterns.
In the last section, Sec.~\ref{sec:conclu} we give a short summary, highlighting important results and conclude with an outlook of possible research applications.

\section{Simulation method}
\label{sec:method}
We perform numerical simulations of the thin film equation (TFE),  
\begin{equation}\label{eq:thinsolve}
     \partial_t h(\mathbf{x},t) = \nabla\cdot\left(M_{\delta}(h)\nabla p\right),
\end{equation}
where $\mathbf{x} = (x,y)$ and $\nabla = (\partial_x, \partial_y)$, by means of a recently developed lattice Boltzmann method code called Swalbe~\cite{zitzLatticeBoltzmannMethod2019, zitzLatticeBoltzmannSimulations2021, zitzSwalbeJlLattice2022, zitzControllingDewettingMorphologies2023}. 
The mobility function $M_{\delta}(h) = \frac{h^2}{\mu\alpha_{\delta}(h)}$ with 
\begin{equation}\label{eq:alphafric}
    \alpha_{\delta}(h) = \frac{6h}{(2 h^2 + 6 \delta h + 3 \delta^2)},
\end{equation}
which for the no-slip boundary condition $(\delta \rightarrow 0)$ becomes $M_{0}(h) = h^3/3\mu$, where $\delta$ is an effective slip length and $\mu$ is the dynamic viscosity, both values can be found in App.~\ref{app:numerics}.
We like to point out, however, that the slip length value is within the weak/intermediate slip regime~\cite{peschkaSignaturesSlipDewetting2019,fetzerQuantifyingHydrodynamicSlip2007, munchLubricationModelsSmall2005} and has been used in previous work~\cite{zitzControllingDewettingMorphologies2023}.
The pressure $p$ in Eq.~(\ref{eq:thinsolve}) is given by,
\begin{equation}\label{eq:filmpressure}
    p = - \gamma\nabla^2 h -\Pi(h),
\end{equation}
with $\Delta h$ being the 2D Laplacian of the liquid-gas interface and $\Pi(h)$ is a so-called disjoining pressure~\cite{schwartzSimulationDropletMotion1998, crasterDynamicsStabilityThin2009, nguyenCompetitionCollapseBreakup2012, gonzalezStabilityLiquidRing2013}
\begin{equation}\label{eq:disjoinpressure}
    \Pi(h,\theta) = \frac{2\gamma}{h_{\ast}}[1-\cos\theta(\mathbf{x})]\left[\left(\frac{h_*}{h}\right)^3 -\left(\frac{h_*}{h}\right)^2\right],
\end{equation}
where $\gamma$ is the surface tension, $h_{\ast}$ is a precursor thickness, see App.~\ref{app:numerics}, at which $\Pi(h_{\ast}, \theta) = 0$ and $\theta$ is an equilibrium contact angle.
By allowing spatial variation of $\theta$ in Eq.~(\ref{eq:disjoinpressure}), we have an effective model for a patterned substrate, see e.g., Refs~\cite{zitzLatticeBoltzmannSimulations2021, zitzControllingDewettingMorphologies2023}. 
The contact angle, in agreement with the lubrication approximation~\cite{oronLongscaleEvolutionThin1997, crasterDynamicsStabilityThin2009}, is set within the bounds $[\pi/18, 2\pi/9]$, except for the banded pattern, see Sec.~\ref{subsubsec:banded}. 

\subsection{Initial conditions}
\begin{figure}
\centering
  \includegraphics[width=0.45\textwidth]{New_fig1.png}
  \caption{Schematic setup of our initial conditions. In (a) we show a render of the initial fluid state and display $R_0$ and $r_0$. 
  Below in panels (b)-(d) we show different wettability patterns with yellow being more wettable than red.
  While (b) depicts the band pattern, (c) and (d) show radial gradient wettabilities.} 
  \label{fig:ringschema}
\end{figure}

We initialise the thickness profile $h_0(\mathbf{x})=h(\mathbf{x}, t=0)$ by first imposing the shape of a toroidal cap, with radial symmetry along the $z$-axis, centred in the origin of the coordinate system and with major an minor radii $R_0$ and $r_0$, whose equation in polar coordinates $(\xi, \phi)$ (with $x = \xi \cos(\phi)$, $y=\xi \sin(\phi)$ reads,
\begin{equation}\label{eq:torus}
h_0(\mathbf{x})=\left(\sqrt{r_0^2 - \left(R_0-\xi\right)^2} - r_0\cos \theta_0 + h_{\ast}\right)
\end{equation}
for $|R_0-\xi|<r_0 \sin \theta_0$ (and $h_0(\mathbf{x})=h_{\ast}$, otherwise); then, we let it relax to the actual equilibrium shape~\cite{diezBreakupFluidRivulets2009} which we slightly perturb with a Gaussian noise with zero mean and variance $10^{-4}r_0^2 \sin^2\theta_0$. 

\section{Results}
\label{sec:dynamics}
\begin{figure*}
    \centering
    \includegraphics[width=0.95\textwidth]{assets/heatcirc_2.png}
    \caption{(a) Heatmap of the film thickness at $t=1.68\tau_m$ for $\psi_0 = 0.21$. 
    All length scales are normalized by $H_D = 25.67$. 
    The red ring in the middle of the rivulet is depicted as blue line plot in (b).
    In (b) we show circular profiles of the film thickness, $h(\xi=R(t),\phi,t)$, for different time steps.
    The dynamics initially selects an unstable mode with wavelength $\lambda \approx (2 \pi R_0)/8$, 
    but as the ring srinks, eventually it breaks up into only $3$ droplets, corresponding to the three maxima of the purple curve.}
    \label{fig:ThreeDToOneD}
\end{figure*}

\subsection{Instability paths}\label{subsec:stability}
Following Gonz\'alez et al.~\cite{gonzalezStabilityLiquidRing2013}, we introduce the dimensionless width parameter, or aspect-ratio, defined as the ratio of the initial rivulet width over radius, $\psi_0 \equiv 2r_0 \sin(\theta_0)/R_0$. 
It was shown numerically and proved theoretically~\cite{gonzalezStabilityLiquidRing2013,nguyenCompetitionCollapseBreakup2012} that if $\psi_0$ is large enough, retractive collapse takes place, otherwise the rivulet tends to breakup and form droplets.

In order to have a quantitative insight on the ring rivulet evolution we track the growth of the following observable:
\begin{equation}\label{eq:delta-h-measure}
       \Delta h(t) = \max_{\phi}h(R(t),\phi,t) - \min_{\phi}h(R(t),\phi,t),
\end{equation}
where $R(t) = (r_i(t)+r_o(t))/2$ and $r_{i,o}(t)$ indicate the location of the inner and outer contact lines, respectively. 

\begin{figure}
    \centering
    \includegraphics[width=0.45\textwidth]{assets/growth-breakup.pdf}
    \caption{Thickness difference $\Delta h$ normalized by the height of a spherical cap droplet that contains all liquid $H_D$ along the curve at radius $\xi(h_{\max})$ over time normalized by $\tau_m \propto (\mu R/\gamma) \sin(\theta_0)/\theta_0^{3/2}$ for the uniform pattern. 
    Different symbols depict different initial conditions. 
    Full or empty symbols distinguish between breakup and collapse. 
    The black dashed curve shows an exponential growth $e^{t/\tau_m}$, see Eq.~(\ref{eq:growth-sigma}).
    }
    \label{fig:first_growth}
\end{figure}
This quantity should grow exponentially in the linear instability regime~\cite{wuBreakupPatternedNanoscale2010, gonzalezStabilityLiquidRing2013, nguyenCompetitionCollapseBreakup2012}, with a growth rate that depends both on the aspect ratio, $\psi_0$, and on the contact angle, $\theta_0$~\cite{gonzalezStabilityLiquidRing2013}; for $\psi_0 \ll 1$, however, the dynamics approximate the linear rivulet case and the growth is essentially independent of $\psi_0$. 
In fact, in this limit, as shown with full symbols in Fig.~\ref{fig:first_growth}, we could achieve a nice collapse of the $\Delta h$ vs. $t$ curves for different couples $(\theta_0,\psi_0)$, provided that the time is rescaled by the characteristic, contact angle dependent time $\tau_m \propto (\mu R/\gamma) \sin(\theta_0)/\theta_0^{3/2}$. 
The rescaling, on the contrary, does not work for large $\psi_0$ (empty symbols in Fig.~\ref{fig:first_growth}), where ring collapse occurs, leading to the closure of the inner hole and to the formation of a single droplet.

\subsection{Dewetting on uniform substrates}\label{subsec:drop-counting}
We start our analysis of the dewetting dynamics of a ring-rivulet considering substrates with uniform wettability (i.e. constant contact angle $\theta(\mathbf{x}) = \theta_0$, see Fig.~\ref{fig:ringschema}(a)). 
In order to discriminate between the two available dewetting paths, we look at the final number of droplets $n_d$, into which the ring rivulet forms as a function of $\psi_0$.
\begin{figure}
    \centering
    \includegraphics[width=0.45\textwidth]{assets/Ndrops_uni_new.pdf}   
    \caption{Droplet number $n_d$ vs aspect ratio $\psi_0$ on uniform substrates for various contact angles $\theta_0$. 
    The solid line depicts the approximated LSA prediction, Eq.~(\ref{eq:maxDrops})~\cite{gonzalezStabilityLiquidRing2013}.}
    \label{fig:max_drops}
\end{figure}
Fig.~\ref{fig:max_drops} shows $n_d$ vs. $\psi_0$ for various contact angles. For small aspect-ratio, $\psi_0 <0.2$, $n_d>1$, indicating that the rivulet undergoes a breakup, whereas for $\psi_0>0.2$ we get $n_d=1$, indicating the retractive collapse to a single central droplet.
When breakup takes place, moreover, we observe that $n_d$ decreases with $\psi_0$ in agreement with the approximated formula 
\begin{equation}\label{eq:maxDrops}
    n_d \approx \frac{\pi}{2\psi_0},
\end{equation}
(solid line in Fig.~\ref{fig:max_drops}) derived from linear stability analysis (LSA) assuming that the most unstable mode
determines the number of droplets~\cite{gonzalezStabilityLiquidRing2013}.
The agreement is particularly good for small contact angles, as expected since the theory leading to (\ref{eq:maxDrops}) does not account for the disjoining pressure~\cite{gonzalezStabilityLiquidRing2013}. 
Interestingly, as the contact angle increases, the number of droplets becomes almost independent of $\psi_0$,
suggesting that the breakup process is mainly driven by the reduced wettability of the substrate.
%%%%%%%%%%%%%%%%%%%%
Taking $\psi_0 \approx 0.2$, at which $n_d$ drops to one, then, as the value discriminating between breakup and collapse, we now focus on the characteristic times of both processes.
Hereafter times are made dimensionless by a characteristic capillary time $t_c$, defined as 
$t_c = \mu r_0/\gamma$.
In Fig.~\ref{fig:breakuptimes} we report the breakup times as a function of $\psi_0$.
The breakup time, $\tau_b$, is defined as the earliest instant of time at
which the line $h(R(t),\phi,t)$ "touches" the substrate, namely
\begin{equation}\label{eq:breakuptime}
\tau_b = \min_t \{t | h(\xi=R(t),\phi,t) = h_{\ast}\}.
\end{equation}
\begin{figure}
    \centering
    \includegraphics[width = 0.45\textwidth]{taub_vs_psi0.pdf}
    \caption{Rivulet breakup times (in units of $t_c = \mu r_0/\gamma$) for various substrate contact angle 
    as a function of the initial aspect-ratio $\psi_0$.}
    \label{fig:breakuptimes}
\end{figure}
For the lowest contact angle, $\tau_b$ grows with $\psi_0$. Such behaviour
was indeed predicted by the theoretical approach of Gonz\'alez et al.~\cite{gonzalezStabilityLiquidRing2013}, assuming that the growth rate of the instability in the linear regime determines also the time scales of breakup. For larger contact angle ($\theta_0 > 10^{\circ}$), i.e. when disjoining pressure effects start to be relevant, instead, the breakup times tend to become almost independent of $\psi_0$, suggesting that the time scales are dictated essentially by the substrate wettability.\\
\begin{figure}
    \centering
    \includegraphics[width = 0.45\textwidth]{tauc_vs_psi0.pdf}
    \caption{Log-log plot of the rivulet collapse times (in units of $t_c = \mu r_0/\gamma$) for various substrate contact angle 
    as a function of the initial aspect-ratio $\psi_0$; the solid line indicates the power law $\tau_c \theta_0^2/t_c \sim \psi_0^{-2}$ (see Eq.(\ref{eq:modeltauc}) and related discussion for further details).}
    \label{fig:collapsetimes}
\end{figure}
Fig.~\ref{fig:collapsetimes} shows the time $\tau_c$ taken by the liquid to fully wet the hole delimited by the ring rivulet which collapses into a single droplet (hence the name "collapse time"), i.e. 
\begin{equation}\label{eq:collapsetime}
\tau_c = \min_t \{t | h(\xi=0,\phi,t) > h_{\ast}\}.
\end{equation}
Upon rescaling by $t_c$ and multiplying by $\theta_0^2$ we observe that the collapse times 
decay with the initial aspect-ratio as $\tau_c \theta_0^2/t_c \sim \psi_0^{-2}$; such behaviour is explained in what follows.
The collapse time can be seen as the time the point (in the radial coordinate) $\xi_1(t) = R(t) - w(t) \approx R(t)$\footnote{This approximation holds since $R \gg w$ most of the time during the retraction process} needs to reach the origin $\xi = 0$. 
We prove in the Appendix (see section \ref{sec:derivation}) that, under certain assumptions, the 
following differential equation for $R(t)$ holds:
\begin{equation}\label{eq:modelC4}
-\frac{d R^2}{dt} \approx 3 \, \theta_0^4 \, \frac{r_0^2}{t_c}.
\end{equation}
The latter equation can be integrated to $R_0^2 - R^2(t) \approx 3 \theta_0^4 r_0^2 t/t_c$.
The collapse time, then, is such that $R^2(\tau_c) \approx 0$, whence 
$\tau_c  \approx R_0^2 t_c/(3 r_0^2 \theta_0^4)$, which can be recast into 
(recall that $\psi_0 \approx 2 r_0 \theta_0/R_0$ for small $\theta$) 
\begin{equation}\label{eq:modeltauc}
\frac{\tau_c}{t_c} \sim  (\theta_0 \psi_0)^{-2},
\end{equation}
shown with a solid line in Fig.~\ref{fig:collapsetimes}.

\subsection{Wettability patterns}\label{subsec:wettability}
We now focus on the rivulet stability and dewetting morphology on 
substrates with space-varying contact angle (i.e. {\it patterned substrates}).
As anticipated in the introduction, we consider two different wettability patterns: i) an annular band 
\begin{equation}\label{eq:theta_band}
    \theta(\xi) =\begin{cases}
        \theta_a \equiv \theta_0,\quad \text{for}~|R_0-\xi| < r_0\sin(\theta_0) \\
        \theta_b,\quad \text{otherwise}
    \end{cases},
\end{equation}
ii) an axially symmetric linear contact angle profile
\begin{equation}\label{eq:theta_grad}
    \theta(\xi) = \frac{\theta_{a}-\theta_{b}}{R_0} \xi + \theta_{b},
\end{equation}
where either $\theta_{a} > \theta_{b}$ (outward pointing gradient) or $\theta_a < \theta_b$ (inward pointing
 gradient).
The different contact angle patterns are then used in Eq.~(\ref{eq:disjoinpressure}) which, in turn, enters in Eq.~(\ref{eq:thinsolve}) through the total pressure, Eq.~(\ref{eq:filmpressure}).
The idea behind such choices is that the former serves as an effective boundary removing the collapse mode, whereas the latter illuminates the force balance between wetting and retraction of the ring-rivulet, see Eqs.~(\ref{eq:theta_band}-\ref{eq:theta_grad}).

\subsubsection{Annular band}\label{subsubsec:banded}
This pattern, Eq.~(\ref{eq:theta_band}), is realized by using $\theta_a \equiv \theta_0 \in [10^{\circ}, 20^{\circ}, 30^{\circ}, 40^{\circ}]$ and $\theta_b = 60^{\circ}$, thus
the ring rivulet is confined on the annular band and the inner hole 
essentially represents an effective energetic barrier for the collapse to take place.
%While this choice seem to contracting the long wave approximation, we actually do not see the film penetrating the %$\theta_b$ region in our numerical experiments.
%Thus, the substrate area $\pm\delta\xi$ away from the ring-rivulets base can be understood as an %effective barrier. 
This setup, somehow, echoes the experiments by 
Edwards et al.~\cite{edwardsControllingBreakupToroidal2021}, where they managed to remove the collapse mode by means of a suitable electrowetting-based manipulation of the substrate.
This can be seen clearly in Fig.~\ref{fig:max_drops_band}, where we report the number of droplets formed upon 
dewetting, noticing that $n_d>1$ for every $\psi_0$ and contact angle. At odds with the uniform case 
(see Fig.~\ref{fig:max_drops}), moreover, the data correlates less with the LSA prediction $n_d = \pi/(2\psi_0)$, except for very small contact angle. For $\psi \gtrsim 0.1$, $n_d$ displays a sort of step-wise dependence on
$\psi_0$. It appears that, by disentangling the radial dynamics driven by contact angle curvature imbalance, the pattern allows for the formation of a larger number of droplets even for relatively larger values of the aspect-ratio (we observe still $n_d \sim O(10)$ up to $\psi_0 \approx 0.2$).
\begin{figure}
    \centering
    %\includegraphics[width=0.45\textwidth]{n_drops_vs_psi_uniform.eps}
    %\includegraphics[width=0.45\textwidth]{n_drops_vs_psi_uniform.eps}
    \includegraphics[width=0.45\textwidth]{assets/Ndrops_ban_new.pdf}    
    \caption{Droplet number $n_d$ vs aspect ratio $\psi_0$ on substrates patterned with 
    the annular band, Eq.(\ref{eq:theta_band}).
    for various contact angles $\theta_0$. The solid line depicts the 
    approximated LSA prediction, Eq.~(\ref{eq:maxDrops})~\cite{gonzalezStabilityLiquidRing2013}.}
    \label{fig:max_drops_band}
\end{figure}
\begin{figure}
    \centering
    \includegraphics[width=0.48\textwidth]{taub_vs_psi0_pattern_annular.pdf}
    \caption{Log-log plot of the ring rivulet breakup times as a function of the initial aspect-ratio, for the annular band pattern, Eq.~\ref{eq:theta_band}.
    Different colors depict different contact angle $\theta_a$ ($\theta_b = 60^{\circ}$ is kept fixed).
    The linear function of $\psi_0$ is reported with a dashed line as a guide for the eye.
    }
    \label{fig:bandBreakupT}
\end{figure}
%Because the initial configuration is unstable, the ring-rivulet will eventually rupture and form %droplets, this however can be on long time scales depending on $\psi_0$.
In Fig.~\ref{fig:bandBreakupT}, we show the dependence of the breakup times on $\psi_0$ for different contact angle contrasts. For large $\theta_a$, i.e. small contact angle contrast, where the constraint on the band is less effective, the breakup time grows with $\psi_0$, in agreement with the prediction of the LSA~\cite{gonzalezStabilityLiquidRing2013}. As $\theta_a$ decreases, though, the confining effect decouples radial dynamics and dewetting and $\tau_b$ tends to become almost independent of $\psi_0$, but determined, instead, by the local contact angle. The breakup times are larger than in the uniform substrate case (cf. Fig~\ref{fig:breakuptimes}): so, overall the annular band pattern makes the ring rivulet slightly more stable against rupture.
{\bf ANDREA: This is to be discussed.
Although having numerical experiments run up to $t = 25\tau_m$ we do not always observe the breakup. 
This is indicated in by the $n_{\max} = 1$ data in Fig.~\ref{fig:max_drops}, where the bullets (\textcolor{black}{$\bullet$}) show data from the banded pattern. 
In those simulations the ring-rivulet has not ruptured at the end of the simulation time.
However, $\Delta h$ measurements show growth in some cases.}
% In contrast to the uniform substrate we do not see any $n_{\max} = 0$ event. 
We see that the predicted number of droplets Eq.~(\ref{eq:maxDrops}) does not fit the data as good as for the uniform substrate. 
Similarly to the uniform substrate and in agreement with previous work~\cite{gonzalezStabilityLiquidRing2013} we see similar behavior as in Fig.~\ref{fig:ThreeDToOneD}.
Thus, having $n = n_{\max}$ for early time scales but at rupture we get $n < n_{\max}$.

\subsubsection{Linear radial profile}\label{subsubsec:linwettgrad}
\begin{figure}
    \centering
    \includegraphics[width=0.48\textwidth]{assets/grad_heatmap.pdf}
    \caption{Evolution of the radius of a ring-rivulet $R(t)$ normalized by the droplet height $H_D$ for different contact angle cases (line styles and colours) and $\psi = 0.133$.
    The first argument of $g(a,b)$ is the contact angle in the centre of the numerical domain and the second argument is the contact angle at $R_0$ and beyond. 
    Besides the lines we add heatmap snapshots of the film thickness $h(\mathbf{x},t)$ where yellow indicates a large value and dark blue a small value (colorbars are not scaled).}
    \label{fig:negativewetgrad}
\end{figure}
We consider, here, first a positive contact angle gradient, namely
$\theta_a > \theta_b=40^{\circ}$ in Eq.~(\ref{eq:theta_grad}) and the wettability increases towards the centre of the domain. In other words, the pattern should favour ring retraction and hole collapse. 
Fig.~\ref{fig:negativewetgrad} shows measurements of the ring radius $R(t)$ for different wettability scenarios at $\psi_0 \approx 0.13$: a uniform substrate with   $\theta_0 = 40^{\circ}$, an annular band with $\theta_a =?$ ($\theta_b =?$) and three different linear contact angle profiles with $\theta_a = 10^{\circ},  $
Fig.~\ref{fig:negativewetgrad} shows measurements of the ring radius $R(t)$ for different wettability scenarios at $\psi_0 \approx 0.13$: a uniform substrate with   $\theta_0 = 40^{\circ}$, an annular band with $\theta_a =?$ ($\theta_b =?$) and three different linear contact angle profiles with $\theta_a = \{10^{\circ},20^{\circ},30^{\circ}\}$.
For the uniform and banded cases (solid blue and dashed orange lines, respectively),
$R(t)$ stays almost constant for the entire simulation, signalling that 
the ring ruptures and forms droplets, which essentially do not slide (or slide very little to a new equilibrium radial position) over the substrate. This is as expected given the low value of the initial aspect-ratio $\psi_0 < 0.2$.
From the remaining curves, on the contrary, we can see the impact of the wettability gradient.  {\bf ANDREA: continue...}
In green we have the highest wettability gradient, starting from $\theta = 40^{\circ}$ at $\xi = R_0$ decreasing to $\theta = 10^{\circ}$ at $\xi = 0$. 
The dynamical evolution of the ring-rivulet is clearly different from the uniform and banded case.
Instead of a breakup we see a constant contraction and the ring remains stable until a single central droplet is formed.
While the radius of that droplet is not zero, we made the arbitrary choice to set $R(t) = 0$ if the liquid-solid area is a disk, in line with the topological change and the switch of the Euler characteristic $\chi$ from 0 to 1.
We see that a wettability gradient can transform an initial condition that would break up into a collapsing one.

Even more interesting are the purple and golden curve in Fig.\ref{fig:negativewetgrad}.
From green to golden the wettability gradient becomes smaller, going from a $30^{\circ}$ difference to a $10^{\circ}$ difference.
The purple is in between the green and the golden with a $20^{\circ}$ difference. 
Similar to the green one we see a linear change in $R(t)$ with a kink shortly before a single droplet is formed.
However, in this case the ring-rivulet is breaking up and forms four intermediate droplets, as shown with heatmaps in Fig.~\ref{fig:negativewetgrad}.
Due to the wettability gradient these droplets are not stationary, but are advected towards the centre of the substrate and form a single droplet on a slightly longer time scale as compared to the green curve.
The golden curve has the smallest wettability gradient which effectively decelerating the dynamics.
Similar to the purple one the rivulet breaks up, but it forms eight droplets instead of four, similar to the uniform substrate.
Again the droplets are advected towards the centre of the substrate, but as the gradient is smaller the radius changes on longer time scales. 
Around $t\approx 32\tau_m$ the two droplets closest to each other coalesce and after $t\approx 35\tau_m$ we only have four droplets.
The final state is again a single central droplet.

\begin{figure}
    \centering
    \includegraphics[width=0.48\textwidth]{assets/radius_time_gradient_positive.pdf}
    \caption{Evolution of the radius of a ring-rivulet $R(t)$ normalized by the droplet height $H_D$ for different contact angle cases (line styles and colours) and $\psi = 0.3$.
    The first argument of $g(a,b)$ is the contact angle in the centre of the numerical domain and the second argument is the contact angle at $R_0$ and beyond.}
    \label{fig:positivewetgrad}
\end{figure}
A wettability gradient, therefore, allows us to alter the time scales $t_b$ and $t_c$.
For the smallest positive gradient (golden curve) we are in fact able to switch between states with eight and four droplets or a single droplet simply by adjusting the radial gradient. 
For completeness we then performed numerical experiments with a negative wettability gradient, thus the centre of the substrate is less wettable than the rest.
In Fig.~\ref{fig:positivewetgrad} we show $R(t)$ data of four simulations with $\psi_0 = 0.3$.
The blue curve depicts the evolution on the uniform substrate, as indicated in Fig.~\ref{fig:max_drops} $\psi_0$ values larger than $0.25$ lead to a collapse.
Unsurprisingly, the banded pattern prohibits this collapse and stabilizes the ring during the duration of the numerical experiment as shown by the orange dashed line.
Additionally, we have two negative gradients shown in green and purple. 
With the larger negative gradient (green curve) the centre of the rivulet is actually pushed away from $R_0$ leading to an increase of radius.
Here we also observe a break up in the long time limit, due to the use of periodic boundary conditions we observe the coalescence of two droplets and thus only have three droplets.
Without periodic boundary conditions we most likely find four droplets which would agree with Eq.~(\ref{eq:maxDrops}).
For the smaller negative gradient, we do not observe such an effect and the data is barely distinguishable from the banded case.
In neither of these two case we see a break up until the last iteration of our simulation.
However, once again a wettability gradient prevents collapse.

\section{Conclusions}\label{sec:conclu}
We have presented numerical experiments of a liquid ring-rivulet on a uniform and patterned substrate. 
The basis of our experiments is the thin film equation with a disjoining pressure model.
For rivulets with small initial ratio of annulus width to major radius, $\psi_0$, we are able to reproduce results of Gonz{\'a}lez et al.~\cite{gonzalezStabilityLiquidRing2013} concerning the most unstable mode and the number of droplets in the final stationary equilibrium state, both for uniform and patterned substrates.
For $\psi_0 > 0.25$ the capillary retraction outpaces the breakup and the rivulete collapses to a single central droplet.
We further found a good analytical approximation for the growth rate on the uniform substrate as shown in Fig.~\ref{fig:first_growth}. 

Our main aim, however, is to address the impact of wettability on the dynamics of the ring-rivulet.
We therefore considered two patterns, one that restricts the rivulet to an annular band which effectively removes the collapse mode, and patterns with a positive and negative wettability gradient towards the central axis.
The banded pattern shows good agreement with the predicted number of droplets in the small $\psi_0$ regime. 
Interestingly, the rupture times seem linearly correlated with $\psi_0$ as shown in Fig.~\ref{fig:bandBreakupT}.

The wettability gradient patterns on the other hand allow us to tune the two competing time scales, namely the breakup time $t_b$ and the collapse time $t_c$.
We show this in Fig.~\ref{fig:negativewetgrad}, where the initial conditions ($\psi \approx 0.13$) favor a breakup of the ring, or $t_b < t_c$.
By applying the positive wettability gradient, we can not only force the rivulet to end up in a collapsed state, but can approach this state via different trajectories. 
Having a strong radial gradient allows to skip the breakup and thus $t_c < t_b$.
If we, however, the gradient, we can transition through different droplet states with either four or eight droplets.
In case of a negative radial wettability gradient, we find that similar to the annular band pattern, collapse is prevented.

As a future perspective of this work, we would like to add thermal fluctuations to the system as some experimental realizations using molten metal.
On the other hand a dynamic pattern may allow for radial breakup as well, because currently we only observed azimuthal breakup.

\section*{Author Contributions}
We strongly encourage authors to include author contributions and recommend using \href{https://casrai.org/credit/}{CRediT} for standardised contribution descriptions. Please refer to our general \href{https://www.rsc.org/journals-books-databases/journal-authors-reviewers/author-responsibilities/}{author guidelines} for more information about authorship.

For footnotes in the main text of the article please number the footnotes to avoid duplicate symbols. \textit{e.g.}\ \texttt{\textbackslash footnote[num]\{your text\}}. The corresponding author $\ast$ counts as footnote 1, ESI as footnote 2, \textit{e.g.}\ if there is no ESI, please start at [num]=[2], if ESI is cited in the title please start at [num]=[3] \textit{etc.} Please also cite the ESI within the main body of the text using \dag. For the reference section, the style file \texttt{rsc.bst} can be used to generate the correct reference style.

\section*{Conflicts of interest}
There are no conflicts to declare.

\section*{Acknowledgements}
S. Z. and J. R. acknowledge the financial support from the Independent Research Fund Denmark through a DFF Sapere Aude Research Leader grant (grant number 9063-00018B).



%%%END OF MAIN TEXT%%%

%The \balance command can be used to balance the columns on the final page if desired. It should be placed anywhere within the first column of the last page.

\balance

%If notes are included in your references you can change the title from 'References' to 'Notes and references' using the following command:
%\renewcommand\refname{Notes and references}

%%%REFERENCES%%%
\bibliography{rsc} %You need to replace "rsc" on this line with the name of your .bib file
\bibliographystyle{rsc} %the RSC's .bst file

\appendix
\section{Numerical model and parameters}\label{app:numerics}
The numerical model is based on a single relaxation (SRT) time lattice Boltzmann method. 
The relaxation time $\tau$ is set to $\tau = 1$ in all our numerical experiments. 
This means that the fluid's kinematic viscosity $\nu$ is set to $\nu = 1/6$ and kept constant for all simulations.  


The surface tension $\gamma$ is set to $\gamma = 10^{-2}$ for all results presented here.
We did in fact simulate with different values of $\gamma$ but found that the effect is an overall rescaling of the time scale, larger $\gamma$ speeds up all dynamics, lower $\gamma$ values slows them down.

Lastly $\alpha_{\delta}(h)$ is a substrate friction term that mimics a slip boundary condition with an effective slip length $\delta$
\begin{equation}\label{eq:alphafric_app}
\alpha_{\delta}(h) = \frac{6h}{(2 h^2 + 6 \delta h + 3 \delta^2)}.
\end{equation}
By introducing a precursor layer $h_{\ast}$ and a slip length $\delta$ we regularize the contact line divergence~\cite{huhHydrodynamicModelSteady1971}. 
The slip length lies within the weak/intermediate slip regime~\cite{peschkaSignaturesSlipDewetting2019,fetzerQuantifyingHydrodynamicSlip2007, munchLubricationModelsSmall2005} and the thickness of the precursor film is set to $h_{\ast} = 0.05$.

By applying this solution for $\mathbf{u}$ to the continuity equation we have
\begin{equation}\label{eq:thinsolve_app}
     \partial_t h(\mathbf{x},t) = \nabla\cdot\left(M_{\delta}(h)\nabla p_{\mbox{\tiny{film}}}\right),
\end{equation}
with the mobility function $M_{\delta}(h) = \frac{h^2}{\mu\alpha_{\delta}(h)}$ which for the no-slip boundary condition $(\delta \rightarrow 0)$ reduces to $M_{0}(h) = h^3/3\mu$.
Without loss of generality we set $\rho_0 = 1$ and thus the dynamic viscosity is $\mu = \rho_0 \nu = 1/6$. 

The numerical domain consists of a square lattice with $512\Delta x$ in both horizontal directions.
We further use biperiodic boundary conditions at the edges of the domain. 

\section{Derivation of equation (\ref{eq:modelC4})}\label{sec:derivation}
Let $\theta_A$ and $\theta_B$ be the contact angles at the advancing and receding contact lines, respectively, we have, by the Cox-Voinov law, that 
\begin{equation}\label{eq:modelC2}
\theta_A^3 - \theta_R^3 \propto Ca = \frac{\mu}{\gamma} U = - \frac{\mu}{\gamma} \dot{R}(t)
\end{equation}
(the minus sign comes from the fact that the ring radius is decreasing), whence, if we write 
$\theta_A = \theta_0 + \delta \theta$ and $\theta_R = \theta_0 - \delta \theta$ 
(with $\delta \theta \ll \theta_{A,R}$), we get
\begin{equation}\label{eq:modelC2}
- \frac{\mu}{\gamma} \dot{R}(t) \approx 6 \theta_0^2 \delta \theta.
\end{equation}
Moreover we know from the static solution for the shape of the ring rivulet that, for small 
angles (i.e. such that one can approximate $\tan \theta \approx \theta$)
$\theta_o \approx f(\psi_0) \theta_i$ ($\theta{i,o}$ being the contact angles at the inner and outer contact lines, respectively), where $f(x)$ is a function that, for small argument, goes as $f(x) \sim 1-x$~\cite{gonzalezStabilityLiquidRing2013}. Assuming this relation valid at any time, we can write, $\theta_R \approx (1-\psi(t))\theta_A$, or also $\delta \theta \approx \psi \theta_0/2$, which, plugged into (\ref{eq:modelC2}) gives
\begin{equation}\label{eq:modelC3}
- \frac{\mu}{\gamma} \dot{R}(t) \approx 3 \theta_0^3 \psi.
\end{equation}
Inserting the expression for the characteristic time $t_c = \mu r_0/\gamma$ and the (instantaneous) 
aspect-ratio $\psi(t) = 2r_0\sin \theta_0/R(t) \approx 2 r_0\theta_0/R(t)$ in (\ref{eq:modelC3}), 
we obtain equation (\ref{eq:modelC4})
\begin{equation}
-R\dot{R} = - 2\frac{d R^2}{d t} \approx 6 \, \theta_0^4 \, \frac{\gamma r_0}{\mu} = 6 \, \theta_0^4 \, \frac{r_0^2}{t_c}.
\end{equation}

\end{document}
